\chapter{Performance Analysis}
\label{chap:performance-analysis}

\section{Introduction}
\label{sec:perf-intro}

% TODO: Write brief intro explaining:
% - Purpose of performance evaluation 
% - Focus on WAMR vs Wasmtime comparison
% - Native as baseline reference
% - Overview of what will be analyzed (timing, memory, correctness)

\section{Experimental Setup}
\label{sec:experimental-setup}

\subsection{Hardware Configuration}
\label{subsec:hardware-config}

% TODO: Describe your Pi + Arduino setup
% - Raspberry Pi specifications (model, OS, architecture)
% - Arduino Uno specifications
% - I2C connection between Pi and Arduino
% - Physical wiring diagram (reference to figure if you make one)
% - Why Arduino was chosen for I2C communication

The experimental setup consists of a Raspberry Pi connected to an Arduino Uno via I2C communication.

\textbf{Raspberry Pi Configuration:}
% TODO: Fill in your specific Pi details
\begin{itemize}
    \item Model: YOUR\_PI\_MODEL
    \item Architecture: aarch64-unknown-linux-musl
    \item Operating System: YOUR\_OS\_VERSION
    \item I2C Interface: Hardware I2C via GPIO pins SPECIFY\_PINS
\end{itemize}

\textbf{Arduino Uno Configuration:}
% TODO: Fill in Arduino details
\begin{itemize}
    \item Model: Arduino Uno
    \item I2C Address: [YOUR\_I2C\_ADDRESS] 
    \item Serial Monitor: Used for correctness verification
    \item Firmware: [DESCRIBE\_YOUR\_ARDUINO\_CODE\_BRIEFLY]
\end{itemize}

% TODO: Add figure of your hardware setup if you have photos
% \begin{figure}[htbp]
% \centering
% \includegraphics[width=0.8\textwidth]{figures/hardware-setup}
% \caption{Raspberry Pi and Arduino I2C Test Setup}
% \label{fig:hardware-setup}
% \end{figure}

\subsection{Benchmark Methodology}
\label{subsec:benchmark-methodology}

The performance evaluation follows a two-stage approach to ensure both correctness and reliable performance measurements.

\subsubsection{Stage 1: Correctness Verification}

% TODO: Explain your correctness verification process
Before conducting performance measurements, correctness verification is performed:
\begin{enumerate}
    \item [DESCRIBE\_VERIFICATION\_PROCESS]
    \item Arduino serial monitor output verification
    \item [WHAT\_DO\_YOU\_CHECK\_SPECIFICALLY?]
    \item [HOW\_DO\_YOU\_VERIFY\_READ/WRITE\_SUCCESS?]
\end{enumerate}

% TODO: Maybe include example serial output or reference to appendix
% \begin{lstlisting}[caption=Example Arduino Serial Output for Verification,label=lst:serial-output]
% // TODO: Include actual serial output from your Arduino
% // showing successful I2C communication
% \end{lstlisting}

\subsubsection{Stage 2: Performance Measurement}

% TODO: Describe your Criterion benchmark setup
After correctness verification, performance measurements are conducted using:
\begin{itemize}
    \item \textbf{Framework:} Criterion.rs statistical benchmarking
    \item \textbf{Output Format:} JSON for automated analysis
    \item \textbf{Measurement Types:} [LIST\_YOUR\_BENCHMARK\_CATEGORIES]
    \item \textbf{Statistical Method:} [DESCRIBE\_CRITERION\_SETTINGS]
\end{itemize}

The ``ping-pong'' operation consists of:
% TODO: Describe your exact ping-pong implementation
\begin{enumerate}
    \item Write [X] bytes to I2C address [ADDRESS]: \texttt{[YOUR\_DATA\_BYTES]}
    \item Read [X] bytes from the same I2C address
    \item [ANY\_VERIFICATION\_STEP?]
\end{enumerate}

\section{Performance Results}
\label{sec:performance-results}

\subsection{Runtime Setup Performance}
\label{subsec:setup-performance}

% TODO: Present your setup timing results
Table~\ref{tab:setup-performance} shows the initialization time for each runtime implementation.

\begin{table}[htbp]
\centering
\caption{Runtime Setup Performance Comparison}
\label{tab:setup-performance}
\begin{tabular}{lrrr}
\toprule
\textbf{Implementation} & \textbf{Mean (ns)} & \textbf{Std Dev (ns)} & \textbf{Relative to Native} \\
\midrule
% TODO: Fill in your actual benchmark results
Native        & [FILL\_NATIVE\_SETUP]    & [FILL\_STD]  & 1.0x \\
WAMR          & [FILL\_WAMR\_SETUP]      & [FILL\_STD]  & [CALC]x \\
Wasmtime      & [FILL\_WASMTIME\_SETUP]  & [FILL\_STD]  & [CALC]x \\
\bottomrule
\end{tabular}
\end{table}

% TODO: Add your analysis of setup performance
% Focus on WAMR vs Wasmtime differences
% Why is Wasmtime so much slower?
% Implications for embedded usage

\subsection{Execution Performance}
\label{subsec:execution-performance}

\subsubsection{Cold Execution}

% TODO: Present cold execution results
Table~\ref{tab:cold-performance} presents the performance of the first execution after runtime setup.

\begin{table}[htbp]
\centering
\caption{Cold Execution Performance}
\label{tab:cold-performance}
\begin{tabular}{lrrr}
\toprule
\textbf{Implementation} & \textbf{Mean (ns)} & \textbf{95\% CI Lower} & \textbf{95\% CI Upper} \\
\midrule
% TODO: Fill in your cold execution data
Native        & [FILL]    & [FILL]  & [FILL] \\
WAMR          & [FILL]    & [FILL]  & [FILL] \\
Wasmtime      & [FILL]    & [FILL]  & [FILL] \\
\bottomrule
\end{tabular}
\end{table}

\subsubsection{Hot Execution}

% TODO: Present hot execution results
Table~\ref{tab:hot-performance} shows steady-state performance after runtime warmup.

\begin{table}[htbp]
\centering
\caption{Hot Execution Performance}
\label{tab:hot-performance}
\begin{tabular}{lrrr}
\toprule
\textbf{Implementation} & \textbf{Mean (ns)} & \textbf{Median (ns)} & \textbf{MAD (ns)} \\
\midrule
% TODO: Fill in your hot execution data
Native        & [FILL]    & [FILL]  & [FILL] \\
WAMR          & [FILL]    & [FILL]  & [FILL] \\
Wasmtime      & [FILL]    & [FILL]  & [FILL] \\
\bottomrule
\end{tabular}
\end{table}

% TODO: Add performance comparison figure
% \begin{figure}[htbp]
% \centering
% \includegraphics[width=0.9\textwidth]{figures/execution-performance-comparison}
% \caption{Execution Performance Comparison: WAMR vs Wasmtime vs Native}
% \label{fig:execution-comparison}
% \end{figure}

\section{Statistical Analysis}
\label{sec:statistical-analysis}

% TODO: Add statistical analysis of your results
\subsection{Significance Testing}

% TODO: Perform and report statistical tests
% Mann-Whitney U test between WAMR and Wasmtime
% Effect size calculations
% Discussion of practical vs statistical significance

\subsection{Performance Variability}

% TODO: Analyze the stability/consistency of your measurements
% Compare variability between implementations
% Discuss implications for real-world usage

% TODO: Reference box plots or violin plots showing distribution
% \begin{figure}[htbp]
% \centering
% \includegraphics[width=0.9\textwidth]{figures/performance-distributions}
% \caption{Performance Distribution Comparison}
% \label{fig:performance-distributions}
% \end{figure}

\section{Memory Usage Analysis}
\label{sec:memory-analysis}

\subsection{DHAT Profiling Setup}

% TODO: Explain your DHAT configuration
Memory profiling was conducted using DHAT (Dynamic Heap Analysis Tool) with the following configuration:
\begin{itemize}
    \item \textbf{Profiling Targets:} [SPECIFY\_WHAT\_YOU\_PROFILED]
    \item \textbf{Measurement Phases:} Runtime setup, ping-pong execution
    \item \textbf{Output Format:} JSON for automated analysis
\end{itemize}

\subsection{Runtime Setup Memory Usage}

% TODO: Present DHAT results for setup phase
Table~\ref{tab:memory-setup} shows memory allocation patterns during runtime initialization.

\begin{table}[htbp]
\centering
\caption{Memory Usage During Runtime Setup}
\label{tab:memory-setup}
\begin{tabular}{lrrr}
\toprule
\textbf{Implementation} & \textbf{Peak Heap (KB)} & \textbf{Total Allocs} & \textbf{Avg Alloc Size (B)} \\
\midrule
% TODO: Fill in your DHAT setup results
Native        & [FILL]    & [FILL]  & [FILL] \\
WAMR          & [FILL]    & [FILL]  & [FILL] \\
Wasmtime      & [FILL]    & [FILL]  & [FILL] \\
\bottomrule
\end{tabular}
\end{table}

\subsection{Execution Memory Usage}

% TODO: Present DHAT results for ping-pong execution
Table~\ref{tab:memory-execution} shows memory usage patterns during ping-pong operations.

\begin{table}[htbp]
\centering
\caption{Memory Usage During Ping-Pong Execution}
\label{tab:memory-execution}
\begin{tabular}{lrrr}
\toprule
\textbf{Implementation} & \textbf{Heap Growth (KB)} & \textbf{Execution Allocs} & \textbf{Memory Lifetime (ms)} \\
\midrule
% TODO: Fill in your DHAT execution results
Native        & [FILL]    & [FILL]  & [FILL] \\
WAMR          & [FILL]    & [FILL]  & [FILL] \\
Wasmtime      & [FILL]    & [FILL]  & [FILL] \\
\bottomrule
\end{tabular}
\end{table}

% TODO: Analyze memory usage differences
% Why does Wasmtime use more memory?
% Implications for embedded systems
% Memory efficiency comparison

\section{WAMR vs Wasmtime Comparison}
\label{sec:wamr-vs-wasmtime}

% TODO: This is your main focus section
% Direct comparison of the two WebAssembly runtimes
% Analyze strengths/weaknesses of each approach

\subsection{Performance Characteristics}

% TODO: Compare performance profiles
\begin{itemize}
    \item \textbf{Setup Performance:} [ANALYZE\_SETUP\_DIFFERENCES]
    \item \textbf{Steady-State Performance:} [ANALYZE\_EXECUTION\_DIFFERENCES] 
    \item \textbf{Memory Efficiency:} [COMPARE\_MEMORY\_USAGE]
    \item \textbf{Performance Consistency:} [COMPARE\_VARIABILITY]
\end{itemize}

\subsection{Architecture Differences}

% TODO: Discuss architectural differences affecting performance
\begin{itemize}
    \item \textbf{WASI Preview 1 vs Preview 2:} [IMPACT\_ON\_PERFORMANCE]
    \item \textbf{Component Model Overhead:} [WASMTIME\_SPECIFIC\_COSTS]
    \item \textbf{Handwritten vs Generated Bindings:} [BINDING\_IMPACT]
    \item \textbf{Runtime Optimizations:} [DIFFERENT\_OPTIMIZATION\_STRATEGIES]
\end{itemize}

\subsection{Embedded Suitability}

% TODO: Evaluate suitability for embedded applications
\begin{itemize}
    \item \textbf{Resource Constraints:} [MEMORY\_CPU\_REQUIREMENTS]
    \item \textbf{Startup Latency:} [IMPACT\_ON\_REAL\_APPLICATIONS]
    \item \textbf{Deterministic Performance:} [REAL\_TIME\_CONSIDERATIONS]
    \item \textbf{Development Experience:} [TOOLING\_DEBUGGING\_SUPPORT]
\end{itemize}

\section{Flame Graph Analysis}
\label{sec:flamegraph-analysis}

% TODO: Analyze your pprof flame graph results
Flame graph analysis provides insights into CPU time distribution during execution.

% TODO: Reference your flame graph figure
% \begin{figure}[htbp]
% \centering
% \includegraphics[width=1.0\textwidth]{figures/flamegraph-comparison}
% \caption{CPU Time Distribution Flame Graph}
% \label{fig:flamegraph}
% \end{figure}

% TODO: Analyze what the flame graph shows
% Where is time spent in each implementation?
% What are the bottlenecks?
% How do call stacks compare between WAMR and Wasmtime?

\subsection{Hotspot Analysis}

% TODO: Identify performance hotspots from flame graph
\begin{itemize}
    \item \textbf{WAMR Hotspots:} [IDENTIFY\_EXPENSIVE\_FUNCTIONS]
    \item \textbf{Wasmtime Hotspots:} [IDENTIFY\_EXPENSIVE\_FUNCTIONS]
    \item \textbf{Common Bottlenecks:} [SHARED\_EXPENSIVE\_OPERATIONS]
\end{itemize}

\subsection{Call Stack Comparison}

% TODO: Compare call stack depth and complexity
% Which runtime has simpler/more complex execution paths?

\section{Summary}
\label{sec:performance-summary}

% TODO: Brief summary of key findings
% Keep this short since you'll have full conclusion chapter
This chapter presented a comprehensive performance analysis comparing WAMR and Wasmtime implementations of the I2C WASI interface.

\textbf{Key Findings:}
\begin{itemize}
    \item [SUMMARIZE\_SETUP\_PERFORMANCE\_DIFFERENCE]
    \item [SUMMARIZE\_EXECUTION\_PERFORMANCE\_SIMILARITY]
    \item [SUMMARIZE\_MEMORY\_USAGE\_DIFFERENCES] 
    \item [SUMMARIZE\_EMBEDDED\_SUITABILITY\_ASSESSMENT]
\end{itemize}

% TODO: Lead into next chapter (Discussion?)
The implications of these findings for embedded WebAssembly applications are discussed in Chapter~\ref{chap:discussion}.