\chapter*{Societal Reflection}
\chaptermark{Societal Reflection}
\addcontentsline{toc}{chapter}{Societal Reflection}

% TODO: Duurzaamheidsreflectie kan bespreken dat het gebruik van wasm oudere hardware terug leven kan geven omdat deze techniek minder kost resources nodig heeft dan andere virtualisatie methoden. Maar uitgebreid onderzoek doen naar resultaten van andere onderzoeken die aanwijzen hoeveel dedicated virtualisatietechnieken voor embedded/contraint devices inhoudt.

This chapter is only for the engineering technology ("industrieel ingenieur") students. It contains a societal reflection of one to three pages. You can interpret this very broadly. For example, this chapter can include one or more of the following.

\begin{itemize}
    \item Placing the work in a broader societal context. For example, how does this work impact or contribute to recent societal change such as digital transformation or the AI revolution?
    \item Explaining how the work contributes to the implementation of the United Nations (UN) Sustainable Development Goals (SDGs). For more information, see \url{https://en.wikipedia.org/wiki/Sustainable_Development_Goals} and \url{https://www.sdgs.be/nl/sdgs}.
    \item Reflecting on the ethical impact of this thesis or the used datasets. For example, how does this impact people's privacy? Is there bias available in the datasets? Could this be used for military or dual-use purposes?
    \item Investigating how the resulting work conforms to relevant laws or technical standards such as the GDPR, AI act and the CRA.
\end{itemize}
