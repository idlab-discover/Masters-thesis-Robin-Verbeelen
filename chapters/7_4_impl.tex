\chapter{Implementation}
\label{chap:implementation}

This chapter presents the implementation of a WASI Preview 1 I2C interface designed specifically for WebAssembly Micro Runtime (WAMR), addressing the critical requirement for embedded device support in the WASI I2C standardization proposal. Building upon Friedrich's foundational work, this implementation establishes a comprehensive Preview 1 setup that enables resource-constrained embedded systems to participate in the I2C standardization effort.

The implementation serves as an exploration of how the WASI I2C interface could be adapted for Preview 1 environments, providing essential groundwork for the proposal's progression to the next standardization phase, which requires demonstrated compatibility with embedded device constraints.

\section{System Overview \& Architecture}
\label{sec:system-overview}

\subsection{Implementation Context and Scope}

The development effort encompasses five core implementations that collectively demonstrate Preview 1 I2C interface feasibility:

\begin{minted}{text}
    wasip1-i2c-lib/          // Manual Preview 1 bindings library
    wasip1-i2c-guest/        // Core WASM module (target: wasm32-wasip1)
    wasip2-i2c-guest/        // WASM component (target: wasm32-wasip2)
    wamr_impl/               // WAMR host runtime implementation
    wasmtime_impl/           // Wasmtime host runtime implementation
\end{minted}

All implementations in this ecosystem, including the Preview 2 component and Wasmtime host, represent original contributions developed specifically for this research. The system design enables direct comparison between Preview 1 and Preview 2 approaches while maintaining functional equivalence in I2C communication capabilities.

\subsection{I2C Interface Implementation: Simplified vs. Official Specification}

This implementation utilizes a simplified version of the WASI I2C interface rather than the complete official specification. The choice to use a reduced interface provides focus on core I2C functionality while demonstrating the essential capabilities needed for I2C communication.

Code Fragments \ref{lst:official-i2c-interface} and \ref{lst:simplified-i2c-interface} showcase the main differences between the official and the simplified interface definition. The complete definition of the interface is available in the proposal's repository. %TODO: refereer naar de repo

\begin{listing}[H]
    \begin{minted}{rust}
    interface i2c {
        // Defined types and error features
        ... 
        
        variant operation {
            read(u64),
            write(list<u8>)
        }
    
        resource i2c {
            /// Execute complex multi-operation transactions
            transaction: func(address: address, operations: list<operation>) 
                        -> result<list<list<u8>>, error-code>;
            
            /// Basic read operation
            read: func(address: address, len: u64) -> result<list<u8>, error-code>;
            
            /// Basic write operation  
            write: func(address: address, data: list<u8>) -> result<_, error-code>;
            
            /// Combined write-then-read operation
            write-read: func(address: address, write: list<u8>, read-len: u64) 
                       -> result<list<u8>, error-code>;
        }
    }
    \end{minted}
    \caption{Official WASI I2C interface specification with comprehensive transaction support}
    \label{lst:official-i2c-interface}
\end{listing}

\begin{listing}[H]
    \begin{minted}{rust}
    interface i2c {
        // Defined types and error features
        ... 
    
        resource i2c {
            /// Core read functionality
            read: func(address: address, len: u64) -> result<list<u8>, error-code>;
            
            /// Core write functionality
            write: func(address: address, data: list<u8>) -> result<_, error-code>;
        }
    }
    \end{minted}
    \caption{Simplified I2C interface providing core read and write functionality}
    \label{lst:simplified-i2c-interface}
\end{listing}

The simplified interface excludes the transaction and write-read methods, concentrating on the fundamental read and write operations that form the basis of I2C communication. This approach provides sufficient functionality for validating the Preview 1 adaptation approach while reducing implementation complexity. The error handling remains comprehensive, preserving all essential I2C protocol error conditions and their semantic information.

\subsection{Pingpong World and WIT Structure Organization}

The pingpong world serves as the primary application interface that demonstrates the use of the (simplified) I2C interface. This world defines the contract between guest components and their host environment, establishing the essential imports and exports needed for I2C communication validation.

\begin{listing}[H]
    \begin{minted}{rust}
    // Pingpong world using simplified I2C interface
    package my:pingpong;
    
    world pingpong {
        import wasi:i2c/i2c;
        use wasi:i2c/i2c.{i2c};
    
        import get-i2c-bus: func() -> i2c;
        export run: func();
    }
    \end{minted}
    \caption{Pingpong world definition demonstrating I2C resource acquisition and application entry point}
    \label{lst:pingpong-world}
\end{listing}

The pingpong world imports the simplified I2C interface and provides a `get-i2c-bus` function that returns an I2C resource. The world exports a single `run` function that serves as the application's entry point, enabling straightforward validation of the complete I2C communication cycle.

The WIT directory structure follows the standard organization pattern established by the component model ecosystem:

\begin{minted}{text}
wit/
    pingpong.wit              // Main world definition
    deps/
        wasi-i2c/
            i2c.wit           // Simplified I2C interface
            world.wit         // I2C world imports
\end{minted}

The `deps/wasi-i2c/` directory contains the simplified I2C interface definition along with its world imports. The simplified version has no need for the additional delay.wit and is therefore removed from the directory and the world.wit. This structure enables both guest and host implementations to reference the same interface definitions, ensuring consistency across the entire ecosystem.

\subsection{Technical Environment}

The implementation targets the following runtime environment specifications:

\begin{itemize}
    \item \textbf{WAMR Rust SDK Version}: v1.1.0 (uses WAMR Version 2.1.2)
    \item \textbf{Wasmtime Version}: v35.0.0
    \item \textbf{Rust Toolchain}: 1.88.0 with \texttt{wasm32-wasip1} and \texttt{wasm32-wasip2} targets
    \item \textbf{Target Architecture}: ARM64 (aarch64-unknown-linux-musl) for Raspberry Pi deployment
    \item \textbf{Component Tooling}: \texttt{cargo-component} v0.21.1 for Preview 2 builds
\end{itemize}

\subsection{Preview 1 vs Preview 2 Architectural Differences}

The fundamental challenge lies in adapting Preview 2's sophisticated resource-based component model to Preview 1's minimalist function-based interface system. This adaptation requires careful consideration of how high-level abstractions can be preserved while working within Preview 1's limited type system.

\begin{listing}[H]
    \begin{minted}{rust}
    // Preview 2: Resource-based approach with automatic lifecycle management
    world pingpong {
        import wasi:i2c/i2c;
        export run: func();
    }
    
    // Preview 1: Function-based approach requiring manual resource management
    world wasip1-pingpong {
        import host_open: func() -> u32;
        import host_write: func(handle: u32, addr: u16, len: usize, offset: usize) -> u8;
        import host_read: func(handle: u32, addr: u16, len: usize, offset: usize) -> u8;
        import host_close: func(handle: u32);
        export run: func();
    }
    \end{minted}
    \caption{Architectural comparison highlighting semantic gap between Preview versions}
    \label{lst:preview-comparison}
\end{listing}

Preview 2's component model provides automatic resource lifecycle management, type-safe binding generation, and sophisticated error handling mechanisms. In contrast, Preview 1 requires manual implementation of all these features through carefully designed function interfaces and explicit resource tracking systems.

\section{Canonical ABI Adaptation and Design Choices}
\label{sec:canonical-abi-adaptation}

\subsection{Canonical ABI Compatibility and Implementation Options}

The WebAssembly Component Model defines a Canonical Application Binary Interface (ABI) that specifies how WIT definitions are translated to bits and bytes for inter-component communication. This ABI provides standardized mechanisms for passing complex data structures, managing resources, and handling errors between WebAssembly components and their host environments.

\subsection{Context-Driven ABI Design Decisions}

Rather than indicating a limitation or incompatibility, the departure from strict Canonical ABI compliance represents a deliberate design choice based on the specific context and constraints of embedded I2C communication. Understanding the application domain enables targeted optimizations that would not be possible with a generic ABI approach.

\textbf{Error Handling Optimization}: The Canonical ABI represents errors using a discriminated union approach, where the error type discriminant and error-specific data (such as NoAcknowledgeSource) would be stored in separate fields, similar to C union structures. The Preview 1 implementation consolidates this information into a single byte using bit manipulation, providing an errno-like interface that reduces memory usage and simplifies error propagation in resource-constrained environments.

\textbf{Memory Management Adaptation}: While the Canonical ABI specifies host-managed memory allocation as the standard, this implementation makes use of guest-managed allocations to showcase the flexibility of Preview 1.

\subsection{Trade-offs of Context-Specific Optimization}

\textbf{Advantages of Custom Approach}:
\begin{itemize}
    \item \textbf{Embedded Optimization}: Targeted optimizations for I2C communication patterns reduce memory usage and improve predictability for embedded deployment scenarios.
    \item \textbf{Simplified Error Model}: The errno-style error reporting aligns with embedded programming conventions and reduces complexity in error handling logic.
    \item \textbf{Resource Efficiency}: Compact error encoding and optimized memory patterns minimize overhead in resource-constrained environments.
    \item \textbf{Deterministic Behavior}: Context-specific optimizations enable more predictable performance characteristics suitable for real-time embedded applications.
\end{itemize}

\textbf{Disadvantages of Deviation from Standards}:
\begin{itemize}
    \item \textbf{Reduced Interoperability}: Custom ABI choices limit compatibility with generic WebAssembly tooling and component model infrastructure.
    \item \textbf{Development Overhead}: Manual implementation requires more effort compared to automatic Canonical ABI generation.
    \item \textbf{Maintenance Complexity}: Custom ABI implementations require manual updates when interface specifications change.
    \item \textbf{Limited Reusability}: Context-specific optimizations may not transfer to other use cases or interface types.
\end{itemize}

The decision to deviate from the Canonical ABI represents a conscious trade-off between universal compatibility and context-specific optimization. While the Canonical ABI could be fully implemented for this use case, the specific constraints and patterns of embedded I2C communication justify targeted optimizations that provide benefits in the target deployment environment.

\section{WASI Preview 1 Bindings Implementation}
\label{sec:wasip1-bindings}

\subsection{Conditional Compilation Architecture}

The \texttt{wasip1-i2c-lib} crate represents the core innovation of this implementation, providing a carefully architected bridge between the high-level semantics defined in WIT specifications and the low-level function interfaces required by Preview 1. The library employs conditional compilation to support both guest and host use cases without imposing unnecessary dependencies on either environment.

\begin{listing}[H]
    \begin{minted}{toml}
    # wasip1-i2c-lib/Cargo.toml
    [features]
    default = []        # Common type definitions and utility only
    guest-utils = []    # Enables guest-specific resource management functionality
    
    [lib]
    crate-type = ["lib"]
    \end{minted}
    \caption{Feature flag configuration enabling flexible deployment across guest and host environments}
    \label{lst:conditional-compilation}
\end{listing}

The \texttt{guest-utils} feature flag determines whether guest-specific functionality is compiled, allowing the same codebase to serve multiple deployment scenarios. When enabled, the library provides high-level resource management abstractions for guest modules. When disabled, it provides only the type definitions and utility functions needed by host implementations.

\subsection{Type System Design and WIT Compatibility}

The type system design preserves the semantic richness of WIT specifications while adapting to Preview 1's constraints. Each WIT type has a corresponding Preview 1 representation that maintains the same information content but fits within the limitations of primitive parameter passing.

\begin{listing}[H]
    \begin{minted}{rust}
    // wasip1-i2c-lib/src/common.rs
    pub type I2cAddress = u16;
    pub type I2cResourceHandle = u32;
    
    #[repr(u8)]
    #[derive(Debug, Clone, Copy)]
    pub enum NoAcknowledgeSource {
        Address,
        Data,
        Unknown,
    }
    
    #[derive(Debug, Clone)]
    pub enum ErrorCode {
        None,
        Bus,
        ArbitrationLoss,
        NoAcknowledge(NoAcknowledgeSource),
        Overrun,
        Other,
    }
    \end{minted}
    \caption{Type definitions maintaining semantic compatibility with WIT specifications while using primitive representations}
    \label{lst:type-definitions}
\end{listing}

\subsection{Error Encoding and Recovery Mechanisms}

The error handling system represents the most sophisticated aspect of the type translation challenge. WIT's rich error types are encoded into single-byte values for efficiency, using a carefully designed bit-packing scheme that preserves all semantic information. An additional \texttt{ErrorCode::None} was added to support the earlier discussed errno-style reporting for when the call is successful.

\begin{listing}[H]
    \begin{minted}{rust}
    impl ErrorCode {
        /// Encode error into Preview 1 compatible byte representation
        pub fn lower(&self) -> u8 {
            match self {
                ErrorCode::None => 0b000_00000,
                ErrorCode::Bus => 0b001_00000,
                ErrorCode::ArbitrationLoss => 0b010_00000,
                ErrorCode::NoAcknowledge(source) => {
                    let no_ack_bits = source.lower();
                    0b011_00000 | no_ack_bits  // Combine error type with source info
                }
                ErrorCode::Overrun => 0b100_00000,
                ErrorCode::Other => 0b101_00000,
            }
        }
    
        /// Decode Preview 1 byte representation back to rich error type
        pub fn lift(val: u8) -> ErrorCode {
            let error_type = val >> 5;           // Extract first 3 bits
            let error_variant = val & 0b000_11111; // Extract last 5 bits
            
            match error_type {
                0 => ErrorCode::None,
                1 => ErrorCode::Bus,
                2 => ErrorCode::ArbitrationLoss,
                3 => ErrorCode::NoAcknowledge(
                        NoAcknowledgeSource::lift(error_variant)
                    ),
                4 => ErrorCode::Overrun,
                _ => ErrorCode::Other,
            }
        }
    }
    \end{minted}
    \caption{Bidirectional error encoding scheme preserving WIT semantic information within Preview 1 constraints}
    \label{lst:error-encoding}
\end{listing}

This encoding scheme utilizes the upper three bits for error type classification and the lower five bits for error-specific information, ensuring that no semantic information is lost during the Preview 1 translation process.

\subsection{Guest-Side Resource Management}

The guest-side resource management system provides lifecycle management based on \acrfull{raii}. It mirrors the automatic resource handling available in Preview 2 components:

\begin{listing}[H]
    \begin{minted}{rust}
    // wasip1-i2c-lib/src/guest.rs
    #[link(wasm_import_module = "host")]
    unsafe extern "C" {
        #[link_name = "host_open"]
        unsafe fn host_open() -> I2cResourceHandle;
        #[link_name = "host_read"]
        unsafe fn host_read(_: I2cResourceHandle, _: I2cAddress, 
                           _: usize, _: *mut u8) -> u8;
        #[link_name = "host_write"]
        unsafe fn host_write(_: I2cResourceHandle, _: I2cAddress, 
                            _: usize, _: *const u8) -> u8;
        #[link_name = "host_close"]
        unsafe fn host_close(_: I2cResourceHandle);
    }
    
    #[repr(transparent)]
    pub struct I2cResource {
        handle: I2cResourceHandle,
    }

    impl I2cResource {
        /// Create new I2C resource, acquiring handle from host
        pub fn new() -> Self {
            let handle = unsafe { host_open() };
            Self { handle }
        }
    }
    
    impl Drop for I2cResource {
        /// Automatic resource cleanup ensuring no handle leaks
        fn drop(&mut self) {
            unsafe { host_close(self.handle); }
        }
    }
    \end{minted}
    \caption{Foreign function interface declarations and \acrshort{raii}-based resource management implementation providing automatic lifecycle control for I2C handles}
    \label{lst:guest-ffi-declarations}
\end{listing}

\begin{listing}[H]
    \begin{minted}{rust}
    // wasip1-i2c-lib/src/guest.rs
    impl I2cResource {
        /// Perform I2C write operation with comprehensive error handling
        pub fn write(&self, address: I2cAddress, data: &[u8]) -> Result<(), ErrorCode> {
            let result = unsafe {
                host_write(self.handle, address, data.len(), data.as_ptr())
            };
            
            match ErrorCode::lift(result) {
                ErrorCode::None => Ok(()),
                error => Err(error),
            }
        }
    
        /// Perform I2C read operation returning allocated data buffer
        pub fn read(&self, address: I2cAddress, len: usize) 
                   -> Result<alloc_crate::vec::Vec<u8>, ErrorCode> {
            let mut read_buffer: Vec<core::mem::MaybeUninit<u8>> = 
                Vec::with_capacity(len);
    
            let host_res = unsafe {
                read_buffer.set_len(len);
                host_read(self.handle, address, len, 
                         read_buffer.as_mut_ptr() as *mut u8)
            };
    
            match ErrorCode::lift(host_res) {
                ErrorCode::None => Ok(unsafe { core::mem::transmute(read_buffer) }),
                error => Err(error),
            }
        }
    }
    \end{minted}
    \caption{Function bindings required by a I2C resource, with ``\texttt{read}'' showcasing guest-managed heap allocation}
    \label{lst:guest-resource-management}
\end{listing}

The \acrshort{raii} pattern ensures that I2C resources are automatically cleaned up when they go out of scope, preventing resource leaks that could be particularly problematic in embedded environments where resource recovery might be limited.

\section{Guest Implementations}
\label{sec:guest-implementations}

\subsection{Preview 1 Core Module}

The Preview 1 guest implementation (\texttt{wasip1-i2c-guest}) represents a carefully optimized approach designed specifically for resource-constrained embedded environments. This implementation prioritizes minimal binary size and runtime overhead while maintaining full functional compatibility with the I2C interface specification.

\subsubsection{No-Standard Library Configuration and Binary Size Impact}

The decision to use \texttt{\#![no\_std]} fundamentally impacts both the implementation approach and the resulting binary characteristics. This configuration eliminates the standard library's substantial memory footprint and complex runtime dependencies, resulting in significantly smaller WebAssembly modules more suitable for embedded deployment.

\begin{listing}[H]
    \begin{minted}{rust}
    // wasip1-i2c-guest/src/lib.rs (essential structure)
    #![no_std]
    #![no_main]
    
    use wasip1_i2c_lib::{common::I2cAddress, guest::I2cResource};
    
    extern crate alloc;
    use lol_alloc::{AssumeSingleThreaded, FreeListAllocator};
    
    #[global_allocator]
    static ALLOCATOR: AssumeSingleThreaded<FreeListAllocator> = unsafe {
        AssumeSingleThreaded::new(FreeListAllocator::new())
    };
    
    use core::panic::PanicInfo;
    #[panic_handler]
    fn panic(_info: &PanicInfo) -> ! {
        loop {}
    }
    \end{minted}
    \caption{Essential infrastructure for no\_std WebAssembly module requiring explicit allocator and panic handler provision}
    \label{lst:no-std-infrastructure}
\end{listing}

The \texttt{no\_std} configuration requires explicit provision of memory allocation and panic handling mechanisms that are normally provided by the standard library. The choice of \texttt{lol\_alloc} reflects embedded systems requirements where traditional heap allocators may be inappropriate due to their memory overhead or real-time characteristics.

\begin{listing}[H]
    \begin{minted}{rust}
    // wasip1-i2c-guest/src/lib.rs (main functionality)
    #[unsafe(no_mangle)]
    pub extern "C" fn _start() {
        let data = [0x68u8, 0x65, 0x6c, 0x6c, 0x6f]; // ASCII "hello"
        let slave_addr: I2cAddress = 0x0009;
        let device = I2cResource::new();
        let _ = device.write(slave_addr, &data);
        let _ = device.read(slave_addr, data.len());
    }
    \end{minted}
    \caption{Complete I2C ping-pong implementation demonstrating resource creation, bidirectional communication, and automatic cleanup}
    \label{lst:preview1-main-functionality}
\end{listing}

The resulting binary size demonstrates the effectiveness of this approach: the optimized Preview 1 module compiles to 4.8KB, compared to a significantly larger binary of 38KB that results from standard library inclusion.

\subsubsection{Memory Allocation Strategy: Guest-Managed vs Host-Managed Heap}

The implementation utilizes guest-managed heap allocation, representing an experimental departure from WAMR's default configuration and the Canonical ABI's specification for host-managed memory allocation. This design choice was more experimental and allowed for insights in the different WAMR Heap strategies. 

%TODO refereer naar Understand the Wamr heaps
WAMR supports multiple heap management approaches, each with distinct characteristics and use cases:

\begin{itemize}
    \item \textbf{WASI Heap}: Created when modules are built with WASI-LIBC, managed by the wasi-libc memory allocator. Also used by the host to allocate memory in the linear memory of the guest, when the guest module exports malloc/free functions.
    \item \textbf{Host-Managed Heap}: Created and managed by WAMR's embedded memory system (ems) allocator. The host runtime controls all memory allocation decisions and provides memory to the guest on request.
\end{itemize}

The \textbf{Canonical ABI} specification recommends \textbf{host-managed memory allocation} for several reasons: enhanced security through centralized memory control, improved resource monitoring capabilities, and simplified garbage collection integration. Host-managed approaches provide the runtime with complete visibility into memory usage patterns and enable sophisticated memory protection mechanisms.

\textbf{Rationale for Guest-Managed Allocation}: The decision to implement guest-managed heap allocation was driven by the specific characteristics of I2C read operations and experimental objectives rather than inherent performance benefits:

\begin{itemize}
    \item \textbf{Predictable Buffer Sizing}: I2C read operations require the guest to specify the expected number of bytes in advance. Guest-managed allocation allows the buffer to be pre-allocated based on this known size before initiating the host function call.
    \item \textbf{Error Handling Considerations}: When I2C operations fail, guest-managed allocation results in pre-allocated buffers that cannot be used, representing wasted allocation effort. This trade-off was accepted to explore the guest-managed approach.
    \item \textbf{Implementation Simplicity}: For the specific case of I2C communication, guest-managed allocation simplifies the host-side implementation by eliminating the need for complex memory allocation coordination.
\end{itemize}

Contrary to initial assumptions, guest-managed allocation does not necessarily provide performance advantages over host-managed approaches. The choice primarily impacts allocation timing and responsibility rather than fundamental performance characteristics. Host function call overhead remains similar in both approaches, and the memory allocation costs are comparable whether performed by the guest or host allocator.

The experimental nature of this choice provides valuable insights into the flexibility of Preview 1 memory management strategies while maintaining compatibility with the broader I2C interface semantics.

\subsection{Preview 2 Component: Automated Binding Generation}

The Preview 2 implementation (\texttt{wasip2-i2c-guest}) showcases the sophisticated automation capabilities of the component model while serving as a functional reference implementation.

\subsubsection{Component Model Integration and Toolchain Requirements}

The Preview 2 implementation utilizes \texttt{cargo-component}, a specialized build tool that extends Cargo's functionality to support WebAssembly component generation with automatic WIT binding creation. This tool represents a vision of the future where WASI dependencies can be specified directly in Cargo.toml files, similar to how traditional Rust crates are managed today.

\begin{listing}[H]
    \begin{minted}{rust}
    // wasip2-i2c-guest/src/lib.rs
    #[allow(warnings)]
    mod bindings;
    
    use bindings::Guest;
    use crate::bindings::get_i2c_bus;
    
    struct PingPongComponent;
    
    impl Guest for PingPongComponent {
        fn run() {
            let dev = get_i2c_bus();
            let _ = dev.write(0x09, &[0x68, 0x65, 0x6c, 0x6c, 0x6f]);
            let _ = dev.read(0x09, 5);
        }
    }
    
    bindings::export!(PingPongComponent with_types_in bindings);
    \end{minted}
    \caption{Preview 2 component implementation leveraging automatic binding generation via cargo-component toolchain}
    \label{lst:preview2-guest}
\end{listing}

The \texttt{cargo-component} tool simulates a future ecosystem where WebAssembly components can declare their WASI interface dependencies in the same declarative manner as traditional Rust dependencies. This approach eliminates the need for manual binding generation and provides automatic type safety guarantees between component interfaces.

\begin{listing}[H]
\begin{minted}[breaklines]{toml}
# wasip2-i2c-guest/Cargo.toml (component configuration)
[package.metadata.component]
package = "my:pingpong"

[package.metadata.component.target]
path = "../wit"
world = "pingpong"

[package.metadata.component.target.dependencies]
"wasi:i2c" = { path = "../wit/deps/wasi-i2c" }
\end{minted}
\caption{Component metadata enabling automatic WIT binding generation and dependency management through cargo-component}
\label{lst:component-config}
\end{listing}

The component configuration metadata demonstrates how WASI interfaces could be managed as first-class dependencies in the Rust ecosystem. The `package.metadata.component.target.dependencies` section allows developers to specify WASI interfaces in the same way they would specify traditional crate dependencies, with path-based or registry-based resolution. This approach provides a foundation for a future where WASI interfaces are distributed and versioned through package registries, enabling robust dependency management for WebAssembly components.

\subsubsection{Standard Library Dependency Constraints}

Unlike the Preview 1 implementation, Preview 2 components cannot utilize \texttt{\#![no\_std]} configuration due to fundamental dependencies within the \texttt{wit-bindgen} runtime system. The component model's automatic binding generation relies on standard library facilities for memory management, error handling, and runtime reflection capabilities that are incompatible with \texttt{no\_std} environments.

This architectural limitation means that Preview 2 components inherently require larger binary sizes and more sophisticated runtime environments, making them less suitable for the most resource-constrained embedded applications. The trade-off provides significant development productivity improvements and automatic correctness guarantees at the cost of increased resource requirements.

\subsection{Compilation Process and Binary Size Analysis}

The compilation processes for the two approaches reveal fundamental architectural differences:

\begin{listing}[H]
\begin{minted}[breaklines]{bash}
# Preview 1 compilation: produces minimal WebAssembly core module
cargo build --target wasm32-wasip1 --release
# Result: ~4.8KB optimized binary with manual bindings
\end{minted}
\caption{Preview 1 compilation producing resource-efficient core WebAssembly module}
\label{lst:preview1-compilation}
\end{listing}

\begin{listing}[H]
\begin{minted}[breaklines]{bash}
# Preview 2 compilation: produces feature-rich WebAssembly component
cargo component build --target wasm32-wasip2 --release
# Result: ~34.8KB binary including component model infrastructure
\end{minted}
\caption{Preview 2 compilation via cargo-component producing comprehensive WebAssembly component with metadata}
\label{lst:preview2-compilation}
\end{listing}

The seven-fold size difference illustrates the fundamental trade-off between minimalism and automation. While Preview 2 components provide superior development ergonomics and automatic correctness guarantees, Preview 1 modules achieve the minimal resource footprint essential for embedded deployment scenarios.

\section{Host Runtime Integration}
\label{sec:host-runtime-integration}

\subsection{WAMR Implementation: Low-Level Runtime Integration}

The WAMR host implementation represents the most technically challenging aspect of the system due to the fundamental architectural mismatch between Rust's memory safety model and WAMR's C-based runtime architecture. This implementation must carefully navigate the boundary between safe Rust code and unsafe C FFI operations while maintaining both performance and correctness.

\subsubsection{Runtime Initialization and Function Registration}

\begin{listing}[H]
\begin{minted}[breaklines]{rust}
// wamr_impl/src/wamr_manager.rs (structure definition)
pub struct PingPongRunner {
    func: Function,
    instance: Instance,
    _module: Module,
    _runtime: Runtime, // Explicit ordering ensures correct destruction sequence
}
\end{minted}
\caption{WAMR runtime structure with carefully ordered fields for proper resource cleanup sequence}
\label{lst:wamr-structure}
\end{listing}

\begin{listing}[H]
\begin{minted}[breaklines]{rust}
// wamr_impl/src/wamr_manager.rs (initialization implementation)
impl PingPongRunner {
    pub fn new() -> Result<PingPongRunner, RuntimeError> {
        let runtime = Runtime::builder()
            .use_system_allocator()
            .register_host_function("host_read", host_functions::read as *mut c_void)
            .register_host_function("host_write", host_functions::write as *mut c_void)
            .register_host_function("host_open", host_functions::open as *mut c_void)
            .register_host_function("host_close", host_functions::close as *mut c_void)
            .build()?;

        let module = Module::from_file(&runtime, "wasmodules/guestp1.wasm")?;
        let instance = Instance::new(&runtime, &module, 1024 * 64)?;
        let func = Function::find_export_func(&instance, "_start")?;
        
        Ok(PingPongRunner { 
            _runtime: runtime, 
            _module: module, 
            instance, 
            func 
        })
    }

    pub fn pingpong(&self) -> Result<(), RuntimeError> {
        let params: Vec<WasmValue> = vec![];
        self.func.call(&self.instance, &params)?;
        Ok(())
    }
}
\end{minted}
\caption{WAMR runtime initialization implementing explicit host function registration, module instantiation and exported guest function linking}
\label{lst:wamr-initialization}
\end{listing}

The structure field ordering is critical for proper resource cleanup, as Rust drops fields in declaration order for structs. WAMR exhibits undefined behavior if runtime resources are destroyed in incorrect sequence, making this ordering essential for system stability.

\subsubsection{Memory Translation and Safety Boundaries}

\begin{listing}[H]
\begin{minted}[breaklines]{rust}
// wamr_impl/src/host_functions.rs (write function - part 1)
pub extern "C" fn write(
    exec_env: wasm_exec_env_t,
    handle: I2cResourceHandle,
    addr: I2cAddress,
    len: usize,
    buffer_offset: usize
) -> u8 {
    let module_inst = unsafe { wasm_runtime_get_module_inst(exec_env) };
    if module_inst.is_null() {
        eprintln!("Host: Failed to get module instance");
        return ErrorCode::Other.lower();
    }

    // Validate permissions before proceeding with memory operations
    let can_write = {
        let manager = I2C_PERMISSIONS_MANAGER.lock().unwrap();
        match manager.get_permissions(module_inst, handle) {
            Some(permissions) => permissions.can_write,
            None => {
                eprintln!("Host: Handle {} not found for module instance {:p}", 
                         handle, module_inst);
                return ErrorCode::Other.lower();
            }
        }
    };

    if !can_write {
        eprintln!("Host: Access denied - no write permission for handle {}", handle);
        return ErrorCode::Other.lower();
    }
\end{minted}
\caption{Host function implementation demonstrating comprehensive permission validation and security checks}
\label{lst:wamr-host-function-pt1}
\end{listing}

\begin{listing}[H]
    \begin{minted}{rust}
    // wamr_impl/src/host_functions.rs (write function - part 2)
        // Critical memory translation: WebAssembly offset to native pointer
        let native_buffer = unsafe {
            wasm_runtime_addr_app_to_native(module_inst, buffer_offset as u64)
        } as *mut u8;
        
        if native_buffer.is_null() {
            eprintln!("Host: Invalid buffer pointer - memory translation failed");
            return ErrorCode::Other.lower();
        }
    
        // Create safe slice from validated raw pointer
        let data_slice = unsafe { std::slice::from_raw_parts(native_buffer, len) };
        
        // Perform actual I2C hardware operation
        let mut hardware_guard = I2C_HARDWARE_MANAGER.lock().unwrap();
        match hardware_guard.bus.write(addr as u8, data_slice) {
            Ok(_) => ErrorCode::None.lower(),
            Err(err) => {
                eprintln!("Host: I2C write error: {:?}", err);
                ErrorCode::from(err).lower()
            }
        }
    }
    \end{minted}
    \caption{Memory-safe host function implementation performing critical WebAssembly to native pointer translation}
    \label{lst:wamr-host-function-pt2}
\end{listing}

The memory translation process represents a critical safety boundary where WebAssembly's sandboxed linear memory model intersects with native system capabilities. The \texttt{wasm\_runtime\_addr\_app\_to\_native} function performs the essential translation while validating that the provided offset lies within the instantiated guest module's allocated memory space.

\subsubsection{Capability-Based Security and Permission Management}

The implementation includes a sophisticated permission management system designed to demonstrate capability-based security principles while supporting multiple concurrent module instances. This system represents a foundational approach to how fine-grained access control could be implemented in future I2C interface versions.

\begin{listing}[H]
    \begin{minted}{rust}
    // wamr_impl/src/permission_manager.rs (structure definitions)
    #[derive(Clone, Debug)]
    pub struct I2cPermissions {
        pub can_read: bool,
        pub can_write: bool,
        // Future extensions could include:
        // pub allowed_addresses: HashSet<I2cAddress>,
        // pub max_transfer_size: usize,
        // pub rate_limits: RateLimitConfig,
    }
    
    pub struct I2cPermissionsManager {
        instances: HashMap<*const WASMModuleInstanceCommon, 
                          HashMap<I2cResourceHandle, I2cPermissions>>,
        next_handle: I2cResourceHandle,
    }
    \end{minted}
    \caption{Capability-based security infrastructure providing foundation for fine-grained I2C access control}
    \label{lst:permission-structures}
\end{listing}

\begin{listing}[H]
\begin{minted}[breaklines]{rust}
// wamr_impl/src/permission_manager.rs (core functionality)
impl I2cPermissionsManager {
    pub fn open_handle(&mut self, instance: *const WASMModuleInstanceCommon) 
                      -> I2cResourceHandle {
        let new_handle = self.next_handle;
        self.next_handle = self.next_handle.wrapping_add(1);
        
        // Default permissions - could be customized based on module identity
        let permissions = I2cPermissions {
            can_read: true,
            can_write: true,
        };
        
        self.instances
            .entry(instance)
            .or_insert_with(HashMap::new)
            .insert(new_handle, permissions);
            
        println!("Host: Allocated handle {} for module instance {:p}", 
                new_handle, instance);
        new_handle
    }

    pub fn get_permissions(&self, instance: *const WASMModuleInstanceCommon, 
                          handle: I2cResourceHandle) -> Option<&I2cPermissions> {
        self.instances.get(&instance)?.get(&handle)
    }

    pub fn close_instance(&mut self, instance: *const WASMModuleInstanceCommon) {
        if let Some(handles) = self.instances.remove(&instance) {
            println!("Host: Released {} handles for module instance {:p}", 
                    handles.len(), instance);
        }
    }
}
\end{minted}
\caption{Permission management implementation enabling per-instance, per-handle capability control with future extensibility}
\label{lst:permission-implementation}
\end{listing}

This permission system demonstrates the foundation for capability-based security in I2C interfaces. The current implementation provides basic read/write permissions as an example of how access control could be structured. Future implementations could extend this system to support:

\begin{itemize}
    \item \textbf{Address-Specific Permissions}: Restricting access to specific I2C device addresses
    \item \textbf{Transfer Size Limits}: Preventing resource exhaustion through large I2C transactions
    \item \textbf{Rate Limiting}: Controlling the frequency of I2C operations per module
    \item \textbf{Resource Passing}: Enabling secure transfer of I2C handles between components
\end{itemize}

The system also provides the infrastructure necessary for supporting resource passing between WebAssembly modules, a capability that could enable sophisticated component composition patterns in future I2C interface versions. When resources are passed between modules, the permission system could track ownership transfers and ensure that access control policies are maintained across module boundaries.

\textbf{Note}: The current ACL features serve as architectural examples demonstrating how capability-based security could be implemented. They have not been extensively tested in production scenarios and represent a foundation for future security policy development rather than a complete security solution.

\subsection{Wasmtime Implementation: Component Model Integration}

The Wasmtime implementation serves as both a reference point for functional comparison and a demonstration of how the same I2C interface semantics can be expressed through different WebAssembly runtime architectures:

\begin{listing}[H]
\begin{minted}[breaklines]{rust}
// wasmtime_impl/src/wasmtime_manager.rs (I2C interface implementation)
impl wasi::i2c::i2c::Host for HostState {}

impl wasi::i2c::i2c::HostI2c for HostState {
    fn read(&mut self, res: Resource<wasi::i2c::i2c::I2c>, address: wasi::i2c::i2c::Address,
            len: u64) -> Result<Vec<u8>, wasi::i2c::i2c::ErrorCode> {
        let resource_id = res.rep();
        let i2c_res = self.i2c_devices.get(&resource_id)
            .ok_or(wasi::i2c::i2c::ErrorCode::Other)?;

        if !i2c_res.acl.can_read {
            return Err(wasi::i2c::i2c::ErrorCode::Other);
        }

        let mut buffer = vec![0u8; len as usize];
        let mut i2c = I2C_BUS.lock().unwrap();
        let addr_7bit = (address & 0x7f) as u8;

        match i2c.read(addr_7bit, &mut buffer) {
            Ok(_) => Ok(buffer),
            Err(e) => Err(e.into()),
        }
    }

    fn write(&mut self, res: Resource<wasi::i2c::i2c::I2c>, address: wasi::i2c::i2c::Address,
             data: Vec<u8>) -> Result<(), wasi::i2c::i2c::ErrorCode> {
        let resource_id = res.rep();
        let i2c_res = self.i2c_devices.get(&resource_id)
            .ok_or(wasi::i2c::i2c::ErrorCode::Other)?;

        if !i2c_res.acl.can_write {
            return Err(wasi::i2c::i2c::ErrorCode::Other);
        }

        let mut i2c = I2C_BUS.lock().unwrap();
        let addr_7bit = (address & 0x7f) as u8;

        match i2c.write(addr_7bit, &data) {
            Ok(_) => Ok(()),
            Err(e) => Err(e.into()),
        }
    }
}
\end{minted}
\caption{Wasmtime component model implementation leveraging automatic resource management and type-safe binding generation}
\label{lst:wasmtime-implementation}
\end{listing}

The Wasmtime implementation benefits from automatic resource lifecycle management and type-safe binding generation, significantly reducing implementation complexity compared to the manual WAMR approach. However, this convenience comes at the cost of increased runtime requirements that may not be suitable for all embedded deployment scenarios.

\section{The Ping-Pong Routine: Comprehensive System Validation}
\label{sec:ping-pong-routine}

\subsection{Operation Sequence and Resource Lifecycle}

The ping-pong routine represents more than a simple I2C communication test; it serves as a comprehensive validation of the entire WebAssembly I2C ecosystem, exercising every critical component of the system in a minimal, reproducible manner. Importantly, the routine encompasses the complete resource lifecycle, including resource creation, which represents a significant portion of the operational complexity.

\begin{listing}[H]
\begin{minted}[breaklines]{rust}
// Complete ping-pong sequence demonstrating full resource lifecycle
fn complete_pingpong_sequence() -> Result<(), ErrorCode> {
    // Phase 1: Resource Creation and Acquisition
    let device = I2cResource::new();  // Triggers host_open() call, handle allocation
    
    // Phase 2: Write Operation with Memory Transfer
    let message = [0x68, 0x65, 0x6c, 0x6c, 0x6f]; // ASCII "hello"
    device.write(0x09, &message)?;  // Host/guest memory boundary crossing
    
    // Phase 3: Read Operation with Buffer Allocation and Return
    let response = device.read(0x09, 5)?;  // Memory allocation and data transfer
    
    Ok(())
    // Phase 4: Automatic Resource Cleanup (via Drop trait)
}  // device.drop() called here, triggering host_close() and handle deallocation
\end{minted}
\caption{Complete ping-pong implementation demonstrating resource creation, bidirectional I2C communication, and automatic cleanup within a single operation}
\label{lst:complete-pingpong}
\end{listing}

\subsection{System Components Under Test}

The ping-pong routine exercises the following critical system components:

\textbf{Resource Lifecycle Management}: The creation of an \texttt{I2cResource} instance triggers the complete resource acquisition sequence, including host function invocation, handle allocation, permission assignment, and registration in the permission management system. The automatic cleanup through Rust's Drop trait ensures proper resource deallocation and prevents handle leaks.

\textbf{Memory Boundary Transitions}: Both write and read operations require careful translation between WebAssembly linear memory and native memory spaces, testing the memory safety mechanisms implemented in the host functions and validating the pointer translation logic.

\textbf{Error Handling Pathways}: The routine exercises the complete error encoding and decoding pipeline, ensuring that I2C protocol errors are correctly propagated from hardware through the host implementation to the guest application without loss of semantic information.

\textbf{Bidirectional Communication}: The sequence tests both host-to-guest parameter passing (addresses, lengths) and guest-to-host data transfer (message buffers), validating the complete communication infrastructure including memory allocation and data marshalling.

\textbf{Hardware Integration}: The actual I2C communication with the target device ensures that the entire software stack correctly interfaces with real hardware, validating the embedded HAL integration and confirming that the abstraction layers do not interfere with proper I2C protocol implementation.

The implementation successfully demonstrates that WASI Preview 1 provides a viable foundation for I2C communication in resource-constrained WebAssembly environments. The carefully designed binding layer effectively bridges the semantic gap between Preview 1's function-based interface and Preview 2's resource-based component model, enabling embedded systems to participate in the WebAssembly ecosystem while meeting the stringent resource requirements of embedded deployment scenarios.

This work establishes the essential groundwork for the WASI I2C proposal's embedded device support requirements, providing a concrete demonstration that the interface can be successfully adapted to the constraints of resource-limited environments without sacrificing functional capability or semantic correctness.